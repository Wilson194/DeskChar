% options:
% thesis=B bachelor's thesis
% thesis=M master's thesis
% czech thesis in Czech language
% slovak thesis in Slovak language
% english thesis in English language
% hidelinks remove colour boxes around hyperlinks

\documentclass[thesis=B,czech]{FITthesis}[2012/06/26]

\usepackage[utf8]{inputenc} % LaTeX source encoded as UTF-8

\usepackage{graphicx} %graphics files inclusion
% \usepackage{amsmath} %advanced maths
% \usepackage{amssymb} %additional math symbols

\usepackage{dirtree} %directory tree visualisation

% % list of acronyms
% \usepackage[acronym,nonumberlist,toc,numberedsection=autolabel]{glossaries}
% \iflanguage{czech}{\renewcommand*{\acronymname}{Seznam pou{\v z}it{\' y}ch zkratek}}{}
% \makeglossaries

\newcommand{\tg}{\mathop{\mathrm{tg}}} %cesky tangens
\newcommand{\cotg}{\mathop{\mathrm{cotg}}} %cesky cotangens

% % % % % % % % % % % % % % % % % % % % % % % % % % % % % % 
% ODTUD DAL VSE ZMENTE
% % % % % % % % % % % % % % % % % % % % % % % % % % % % % % 

\department{Katedra \ldots (Softwarové inženýrství)}
\title{	MobChar - desktop application for Game master in Dračí doupě}
\authorGN{Jan} %(křestní) jméno (jména) autora
\authorFN{Horáček} %příjmení autora
\authorWithDegrees{Jan Horáček} %jméno autora včetně současných akademických titulů
\supervisor{Ing. Zdeněk Rybola}
\acknowledgements{Poděkování}

\abstractCS{Cílem této práce je usnadnění hráčům Dračího doupěte a~zejména Pánu jeskyně usnadnit vytváření dobrodružství, postav a~předmětů pomocí jednoduché desktopové aplikace. Hlavní výhody aplikace jsou, možnost exportování dobrodružství pro mobilní aplikaci MobChar pro Pána jeskyně nebo exportování do přehledného pdf vhodné pro tisk a hraní Dračího doupěte bez mobilů a jiných zařízení. V práci jsem vytvořil aplikaci, ve které můžete pomocí grafického rozhraní vytvářet mapy dobrodružství propojené s~příšerami a předměty, které se dají na mapě najít. Veškeré výtvory se ukládají pro případné další použití. Můžete si vytvářet sbírku příšer a předmětů, které můžete i~sdílet s~ostatními Pány jeskyně a tak se nechat inspirovat ostatními.  Hlavní přínos této aplikace bude pro hráče Dračího doupěte, kteří by rádi využili moderní technologie pro tvorbu svého dobrodružství. Někteří hráči jistě ocení možnost sdílení dobrodružství s ostatními nadšenci anebo naopak se rádi nechají inspirovat ostatními hráči. Pomocí databáze již dříve vytvořených příšer a předmětů bude velice snadné vytvářet nové a ještě zábavnější dobrodružství.}

\abstractEN{Sem doplňte ekvivalent abstraktu Vaší práce v~angličtině.}

\placeForDeclarationOfAuthenticity{V~Praze}
\declarationOfAuthenticityOption{4} %volba Prohlášení (číslo 1-6)

\keywordsCS{desktopová aplikace, hra na hrdiny, rozšíření, herní scéna, MobChar, Dračí doupě, tvorba dobrodružství, Python, SQLite, XML}

\keywordsEN{desktop application, heroes games, extension, desktop games, MobChar, Dračí doupě, story creation, Python, SQLite, XML}

\begin{document}

% \newacronym{CVUT}{{\v C}VUT}{{\v C}esk{\' e} vysok{\' e} u{\v c}en{\' i} technick{\' e} v Praze}
% \newacronym{FIT}{FIT}{Fakulta informa{\v c}n{\' i}ch technologi{\' i}}

\begin{introduction}
Dračí doupě je populární hra na hrdiny, která vznikla jako česká verze hry Dungeons~\&~dragons. V České Republice si získala velkou oblibu mezi hráči stolních her a počítačových RPG her.\\

Hráči hrají za své hrdiny, které si na začátku dobrodružství vytvořili, putují světem a zažívají rozličná dobrodružství, která pro ně pán jeskyně připraví. Každý hráč si může vybrat z~rozličného výběru ras, které se ve světě vyskytují a zvolit si, jakému povolání se bude věnovat. Ale ať si vybere mocného orkského válečníka, rychlého a úskočného hobitího zloděje nebo moudrého a mocného elfského válečníka, budou ho čekat složitá rozhodnutí, která určí, jakým směrem se dobrodružství bude ubírat dále. \\

Veškerý průběh dobrodružství, ať už se jedná o získání nového obou-ručního meče, naučení nového mocného kouzla nebo postoupení na novou úroveň, je zapotřebí zaznamenávat do takzvaného osobního deníku. Jak bývá zvykem u deskových her, ze kterých hra dračí doupě vlastně vychází, veškeré postupy se zaznamenávají na obyčejný papír, v některých případech předpřipravený formulář. Jak však dobrodružství postupuje, mnohé informace se mění a rozsah informací se stále zvětšuje a je stále více problematické vše zaznamenávat pouze na papíře a udržovat veškeré informace aktuální.\\

Za účelem vyřešit tuto problematiku, vznikla mobilní aplikace MobChar, která umožňuje zapisování veškerých důležitých informací o postavě do mobilního zařízení. Aktualizace všech informací je velice jednoduchá a vše máte hned po ruce. Další velkou výhodou této mobilní aplikace je snadná přenositelnost pro potřeby dalšího pokračování hry. \\

Dračí doupě se lehce lišší od klasických stolních her, kde plán hry nebo dobrodružství máte jeden a toho se musíte držet. Zde celé dobrodružství vymýšlí další hráč, takzvaný pán jeskyně, který při hře celé dobrodružství řídí a popisuje hrdinům. Veškeré rozhodnutí, které neučiní samotní hráči rozhodne pán jeskyně. Celé dobrodružství musí mít předem připravené a nejedná se o pár věcí, ale o hromady ruzných papírů a poznámek, které jsou k dobrodružství důležité. 
\end{introduction}

\chapter{Cíl práce}

\chapter{Analýza a návrh}

\chapter{Realizace}

\begin{conclusion}
	%sem napište závěr Vaší práce
\end{conclusion}

\bibliographystyle{csn690}
\bibliography{mybibliographyfile}

\appendix

\chapter{Seznam použitých zkratek}
% \printglossaries
\begin{description}
	\item[HnH] Hra na hrdiny
	\item[DrD] Dračí doupě
	\item[RPG] Role-playing game
	\item[PJ] Pán jeskyně
	\item[DAO] Data access object	
\end{description}

 

\chapter{Obsah přiloženého CD/USB}

%upravte podle skutecnosti

\begin{figure}
	\dirtree{%
		.1 readme.txt\DTcomment{stručný popis obsahu CD}.
		.1 exe\DTcomment{adresář se spustitelnou formou implementace}.
		.1 src.
		.2 impl\DTcomment{zdrojové kódy implementace}.
		.2 thesis\DTcomment{zdrojová forma práce ve formátu \LaTeX{}}.
		.1 text\DTcomment{text práce}.
		.2 thesis.pdf\DTcomment{text práce ve formátu PDF}.
		.2 thesis.ps\DTcomment{text práce ve formátu PS}.
	}
\end{figure}

\end{document}
